\documentclass[11pt]{article}

% Page layout (single-column, professional margins)
\usepackage[margin=1in]{geometry}

% Fonts and text
\usepackage{times}
\usepackage{latexsym}
\usepackage[T1]{fontenc}
\usepackage[utf8]{inputenc}
\usepackage{microtype}

% Figures and tables
\usepackage{graphicx}
\usepackage{booktabs}
\usepackage{caption}

% Math
\usepackage{amsmath}

% URLs
\usepackage{url}

% Title
\title{Group X Progress Report:\\Heart Disease Prediction using Machine Learning}

\author{
ZhiChong Lin, ZiDi Yao, Ke Ma\\
\texttt{yaoz25@mcmaster.ca, mak11@mcmaster.ca, linz8@mcmaster.ca}
}

\date{}

\begin{document}
\maketitle

% ============================================================
\section{Introduction}
This section introduces the problem and motivation of your project.  
You may adapt the motivation from your original proposal.  
Typical content includes:  
(1) What problem you are solving,  
(2) Why it matters,  
(3) Why machine learning is suitable,  
(4) Your project objective.  
This should be about 0.25–0.5 pages.

% ============================================================
\section{Related Work}
This section summarizes the most relevant previous work.  
If no identical problem exists, describe the most similar tasks such as:  
– Medical risk prediction  
– Heart disease datasets  
– Classic ML models like logistic regression / SVM in healthcare  
Cite at least five references (use custom.bib).  
Length: 0.25–0.5 pages.

% ============================================================
% ============================================================
\section{Dataset and Preprocessing}

In this project, we use the \textit{Heart Failure Prediction Dataset}, 
which contains 918 patient records and 12 columns: 11 input features and one 
binary target label. All column names were normalized by removing whitespace to 
ensure consistency during preprocessing. The dataset includes a variety of 
clinically relevant attributes commonly used in cardiovascular disease prediction. 
A summary of the raw features is provided in Table~\ref{tab:rawfeatures}.

\subsection{Dataset Description}

\begin{table}[htbp]
\centering
\begin{tabular}{ll}
\toprule
\textbf{Feature} & \textbf{Description} \\
\midrule
Age & Age of patient (years) \\
Sex & Biological sex (M/F) \\
ChestPainType & Type of chest pain (ATA, ASY, NAP, TA) \\
RestingBP & Resting blood pressure (mm Hg) \\
Cholesterol & Serum cholesterol (mg/dL) \\
FastingBS & Fasting blood sugar (0 = normal, 1 = high) \\
RestingECG & Resting ECG results (Normal, ST, LVH) \\
MaxHR & Maximum heart rate achieved \\
ExerciseAngina & Exercise-induced angina (Y/N) \\
Oldpeak & ST depression induced by exercise \\
ST\_Slope & Slope of the ST segment (Up, Flat, Down) \\
HeartDisease & Target label (1 = disease, 0 = healthy) \\
\bottomrule
\end{tabular}
\caption{Raw dataset features.}
\label{tab:rawfeatures}
\end{table}

\subsection{Target Extraction}

The target variable \texttt{HeartDisease} is a binary classification label 
indicating whether a patient shows clinical signs of heart disease.  
We extracted this column as a one-dimensional vector and stored it separately 
in \texttt{processed/y.csv}. The label was converted to integer form  
(1 = disease, 0 = no disease) for compatibility with machine learning models.

\subsection{Feature Preprocessing}

The input feature matrix was constructed by removing the target label and 
retaining the remaining 11 columns. Because the dataset includes both numerical 
and categorical variables, several preprocessing steps were required to convert 
all features into a fully numeric representation.

\paragraph{Binary Encoding.}
Three categorical features contain only two possible values and were encoded 
using standard 0/1 mappings:
\begin{itemize}
    \item \textbf{Sex:} M $\rightarrow$ 1,\; F $\rightarrow$ 0
    \item \textbf{ExerciseAngina:} Y $\rightarrow$ 1,\; N $\rightarrow$ 0
    \item \textbf{FastingBS:} preserved as integer 0/1
\end{itemize}

\paragraph{One-Hot Encoding for Multi-Class Features.}
Features with more than two categories were transformed using one-hot encoding:
\begin{itemize}
    \item ChestPainType (4 categories)
    \item RestingECG (3 categories)
    \item ST\_Slope (3 categories)
\end{itemize}
This produced new indicator columns such as 
\texttt{ChestPainType\_ASY}, \texttt{RestingECG\_Normal}, and 
\texttt{ST\_Slope\_Up}.  
All categories were retained (\texttt{drop\_first=False}) to avoid imposing any 
false ordinal relationships among categorical values.

\subsection{Final Processed Dataset}

After preprocessing, the final feature matrix contains 
\textbf{918 samples and 18 fully numeric columns}.  
The processed features were saved to \texttt{processed/X\_encoded.csv}, 
and a list of all generated feature names was stored in 
\texttt{processed/feature\_names.txt}.  
This final dataset serves as the input to all models and experiments 
conducted in the remainder of the project.


\section{Model Inputs (Features)}
Describe the input representation to the model, e.g.:

\begin{itemize}
    \item The 18-dimensional processed feature vector
    \item Whether any normalization was applied
    \item Whether feature engineering or selection was performed
\end{itemize}

This section corresponds to Item 3 in the project instructions.

% ============================================================
\section{Model Implementation}
Describe the machine learning model(s) used.

Examples:
\begin{itemize}
    \item RBF-kernel Support Vector Machine (SVM)
    \item Justification for using SVM
    \item Hyperparameters used (C, gamma)
    \item Training pipeline and libraries (scikit-learn)
\end{itemize}

This section corresponds to Item 4 of the project instructions.

% ============================================================
\section{Evaluation Strategy and Results}

\subsection{Evaluation Method}
Explain why you used stratified K-fold cross validation  
(e.g., small dataset size, need for robust evaluation).

\subsection{Metrics}
Explain why accuracy, precision, recall, F1, and AUC-ROC are important in medical diagnosis.

\subsection{Results}
Insert your figures:

\begin{figure}[htbp]
\centering
\includegraphics[width=0.9\linewidth]{kfoldvalidation.png}
\caption{5-fold precision scores.}
\end{figure}

\begin{figure}[htbp]
\centering
\includegraphics[width=0.9\linewidth]{metrics_output.png}
\caption{Summary metrics across folds.}
\end{figure}

\begin{figure}[htbp]
\centering
\includegraphics[width=0.9\linewidth]{AUC.png}
\caption{AUC–ROC curves.}
\end{figure}

\begin{figure}[htbp]
\centering
\includegraphics[width=0.9\linewidth]{confuse_matrix.png}
\caption{Confusion matrices.}
\end{figure}

% ============================================================
\section{Feedback and Future Plans}
Summarize TA feedback and your improvements:
\begin{itemize}
    \item Replace label encoding with one-hot encoding
    \item Consider switching from SVM to neural networks for performance gains
    \item Create a new GitHub branch for experiments
\end{itemize}

% ============================================================
\section*{Team Contributions}
Describe what each team member worked on.

% ============================================================
% Bibliography
\bibliographystyle{plain}
\bibliography{custom}

\end{document}
