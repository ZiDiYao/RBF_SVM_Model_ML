\documentclass[11pt]{article}

% Page layout (single-column, professional margins)
\usepackage[margin=1in]{geometry}

% Fonts and text
\usepackage{times}
\usepackage{latexsym}
\usepackage[T1]{fontenc}
\usepackage[utf8]{inputenc}
\usepackage{microtype}

% Figures and tables
\usepackage{graphicx}
\usepackage{booktabs}
\usepackage{caption}

% Math
\usepackage{amsmath}

% URLs
\usepackage{url}

% Title
\title{Group X Progress Report:\\Heart Disease Prediction using Machine Learning}

\author{
ZhiChong Lin, ZiDi Yao, Ke Ma\\
\texttt{yaoz25@mcmaster.ca, mak11@mcmaster.ca, linz8@mcmaster.ca}
}

\date{}

\begin{document}
\maketitle

% ============================================================
\section{Introduction}
This section introduces the problem and motivation of your project.  
You may adapt the motivation from your original proposal.  
Typical content includes:  
(1) What problem you are solving,  
(2) Why it matters,  
(3) Why machine learning is suitable,  
(4) Your project objective.  
This should be about 0.25–0.5 pages.

% ============================================================
\section{Related Work}
This section summarizes the most relevant previous work.  
If no identical problem exists, describe the most similar tasks such as:  
– Medical risk prediction  
– Heart disease datasets  
– Classic ML models like logistic regression / SVM in healthcare  
Cite at least five references (use custom.bib).  
Length: 0.25–0.5 pages.

% ============================================================
\section{Dataset and Preprocessing}
Describe the dataset, number of samples, features, data source, and what preprocessing was required.

\subsection{Dataset Description}
Present the raw features in a table:

\begin{table}[h]
\centering
\begin{tabular}{ll}
\toprule
\textbf{Feature} & \textbf{Description} \\
\midrule
Age & Age of patient (years) \\
Sex & Biological sex (M/F) \\
ChestPainType & Chest pain type (ATA, ASY, NAP, TA) \\
RestingBP & Resting blood pressure (mm Hg) \\
Cholesterol & Serum cholesterol (mg/dL) \\
FastingBS & Fasting blood sugar (0/1) \\
RestingECG & ECG results (Normal, ST, LVH) \\
MaxHR & Maximum heart rate achieved \\
ExerciseAngina & Exercise-induced angina (Y/N) \\
Oldpeak & ST depression value \\
ST\_Slope & Slope of ST segment (Up/Flat/Down) \\
HeartDisease & Target label (1 = disease, 0 = healthy) \\
\bottomrule
\end{tabular}
\caption{Raw dataset features.}
\end{table}

\subsection{Target Extraction}
Describe how \texttt{HeartDisease} was extracted as the binary label vector.

\subsection{Feature Preprocessing}
Explain your preprocessing:
\begin{itemize}
    \item Removing whitespace in column names
    \item Encoding binary variables (Sex, ExerciseAngina, FastingBS)
    \item One-hot encoding for multi-class categorical features  
          (ChestPainType, RestingECG, ST\_Slope)
    \item Saved processed data to: \texttt{processed/X\_encoded.csv}
\end{itemize}

\subsection{Final Processed Dataset}
State final shape (e.g., 918 rows × 18 columns) and where it is stored.

% ============================================================
\section{Model Inputs (Features)}
Describe the input representation to the model, e.g.:

\begin{itemize}
    \item The 18-dimensional processed feature vector
    \item Whether any normalization was applied
    \item Whether feature engineering or selection was performed
\end{itemize}

This section corresponds to Item 3 in the project instructions.

% ============================================================
\section{Model Implementation}
Describe the machine learning model(s) used.

Examples:
\begin{itemize}
    \item RBF-kernel Support Vector Machine (SVM)
    \item Justification for using SVM
    \item Hyperparameters used (C, gamma)
    \item Training pipeline and libraries (scikit-learn)
\end{itemize}

This section corresponds to Item 4 of the project instructions.

% ============================================================
\section{Evaluation Strategy and Results}

\subsection{Evaluation Method}
Explain why you used stratified K-fold cross validation  
(e.g., small dataset size, need for robust evaluation).

\subsection{Metrics}
Explain why accuracy, precision, recall, F1, and AUC-ROC are important in medical diagnosis.

\subsection{Results}
Insert your figures:

\begin{figure}[h]
\centering
\includegraphics[width=0.9\linewidth]{kfoldvalidation.png}
\caption{5-fold precision scores.}
\end{figure}

\begin{figure}[h]
\centering
\includegraphics[width=0.9\linewidth]{metrics_output.png}
\caption{Summary metrics across folds.}
\end{figure}

\begin{figure}[h]
\centering
\includegraphics[width=0.9\linewidth]{AUC.png}
\caption{AUC–ROC curves.}
\end{figure}

\begin{figure}[h]
\centering
\includegraphics[width=0.9\linewidth]{confuse_matrix.png}
\caption{Confusion matrices.}
\end{figure}

% ============================================================
\section{Feedback and Future Plans}
Summarize TA feedback and your improvements:
\begin{itemize}
    \item Replace label encoding with one-hot encoding
    \item Consider switching from SVM to neural networks for performance gains
    \item Create a new GitHub branch for experiments
\end{itemize}

% ============================================================
\section*{Team Contributions}
Describe what each team member worked on.

% ============================================================
% Bibliography
\bibliographystyle{plain}
\bibliography{custom}

\end{document}
